\begin{cabstract}
	纳米零价铁(nZVI)越来越多地用于污染土壤和含有氯化溶剂和重金属的地下水的原位修复。但分散在水相中的纳米零价铁颗粒由于磁力作用而具有强烈的聚集趋势,形成远超过微米尺寸的枝状絮体和网状结构,降低了它的有效表面积和反应活性在实验室研究和现场试验中的迁移性和稳定性表现均不理想。因此研究纳米零价铁在多孔介质中的迁移,通过制备改性纳米铁增强其稳定性与分散性,提高其在多孔介质中的迁移距离。并通过填充柱实验,研究硫化型纳米零价铁在多孔介质中的迁移过程以及多孔介质水力特性的变化过程。建立考虑拦截、布朗扩散、重力沉降和多孔介质性质变化影响的非稳态变参数数学模型。通过模型与实验数据的拟合,研究实验过程中各种机理的变化。
	\ckeywords{包覆型硫化型纳米铁;颗粒稳定性;XDLVO;多孔介质;迁移}
\end{cabstract}

\begin{eabstract}
	Nanometer zero-valent iron (nZVI) is increasingly used for in situ remediation of contaminated soils and groundwater containing chlorinated solvents and heavy metals.  However, due to the magnetic force, the zero-valent iron nanoparticles dispersed in the aqueous phase have a strong tendency to aggregate and form dendritic flocs and network structures with a size far larger than micron, which reduce their effective surface area and reactivity. Their mobility and stability are not ideal in laboratory and field tests.  Therefore, the migration of zero-valent iron nanoparticles in porous media was studied, and its stability and dispersion were enhanced by preparing modified iron nanoparticles to improve its migration distance in porous media.  The migration process of sulfurized zero-valent iron nanoparticles in porous media and the change process of hydraulic characteristics of porous media were studied by packed column experiment.  A mathematical model of unsteady variable parameters considering interception, Brownian diffusion, gravity settlement and property change of porous media is established.  By fitting the model with experimental data, the changes of various mechanisms in the experimental process are studied.  
	\ekeywords{Sulfidated nanoscale zero valent iron, Dispersion stability,Extended DLVO, Porous medium, Transport}
\end{eabstract}